Б) Дано множество функций  L  и отображение  $\mathcal{A}:\ \ L\rightarrow L$.
Проведите исследование:
\begin{enumerate}
	\item Проверьте, что  L  является линейным пространством над полем $\mathbb{R}$.
	\item Выберите в нём базис.
	\item Убедитесь, что отображение $\mathcal{A}$ является линейным (оператором).
	\item Решите задачу о диагонализации матрицы линейного оператора $\mathcal{A}$ в выбранном базисе методом спектрального анализа:
	\begin{itemize}
        \item в случае, если $\mathcal{A}$ имеет скалярный тип, для диагонализации используйте собственный базис.
        \item в случае, если $\mathcal{A}$ имеет общий тип, для диагонализации используйте жорданов базис (приведите матрицу в жорданову форму).
    \end{itemize}
\end{enumerate}

\vspace{10mm}

\textbf{Решение.}

\textit{It is empty but you can fill it!}

\textit{Ответ}: \textit{It is empty but you can fill it!}
