Б) Дано множество функций L и отображение  $\mathcal{A}:\ \ L\rightarrow L$.
Проведите исследование:
\begin{enumerate}
    \item Проверьте, что L является линейным пространством над полем $\mathbb{R}$.
    \item Выберите в нём базис.
    \item Убедитесь, что отображение $\mathcal{A}$ является линейным (оператором).
    \item Решите задачу о диагонализации матрицы линейного оператора $\mathcal{A}$ в выбранном базисе методом спектрального анализа:
    \begin{itemize}
        \item в случае, если $\mathcal{A}$ имеет скалярный тип, для диагонализации используйте собственный базис.
        \item в случае, если $\mathcal{A}$ имеет общий тип, для диагонализации используйте жорданов базис (приведите матрицу в жорданову форму).
    \end{itemize}
\end{enumerate}

$L$ – множество многочленов $P\left(x\right)$ степени не выше 2,

$\displaystyle \mathcal{A}\left(P\left(x\right)\right)=\int_{-1}^{1}{K\left(x,y\right)\ P\left(y\right)\ dy}$, где $K\left(x,y\right)=y^2+2x\left(y-1\right)+\left(1-3y^2\right)x^2$.

\vspace{10mm}

\textbf{Решение.}

\begin{enumerate}
    \item $\forall P(x) \ \deg P \leq 2 \ \exists a, b, c \in \Real \quad P(x) = ax^2 + bx + c$

    $\letsymbol$ $(a, b, c)$ - арифметический вектор. Так как мы знаем, что пространство арифметических векторов линейное,
    а многочлен мы можем представить, как вектор коэффициентов (при этом операции между многочленом и векторами будут давать одинаковые результаты),
    то мы можем сказать, что пространство многочленов изоморфно пространству векторов и соответственно линейно.

    \item Так как $L$ изоморфно $\Real^3$, за базис можно взять стандартный базис:

    $\displaystyle \Set{\overrightarrow{i}, \overrightarrow{j}, \overrightarrow{k}} =
    \begin{pmatrix}
        1 & 0 & 0 \\
        0 & 1 & 0 \\
        0 & 0 & 1
    \end{pmatrix} \sim
    \Set{1, t, t^2} = E_3 \subset L$

    \item $\displaystyle \mathcal(A)(P(x)) = \int_{-1}^1 K(x, y) P(y) dy =
    \int_{-1}^1 (y^2+2x(y-1)+(1-3y^2)x^2) (ay^2 + by + c) dy =
    -\frac{2}{15} (a (4 x^2 + 10x - 3) - 5(2bx - 6cx + c))
    -\frac{2}{15} (4ax^2 + x (10a - 10b + 30c) - 3a - 5c)$

    Тогда $\forall x = (a, b, c) \quad \mathcal{A}(x) = (-\frac{8}{15}a, -\frac{4}{3}a + \frac{4}{3}b - 4c, \frac{2}{5}a + \frac{2}{3}c)$

    Так как каждая компонента вектора - многочлен $Q(a, b, c), \deg P = 1$, то по свойствам сложения и
    умножения $\displaystyle \mathcal{A}(x + y) = \mathcal{A}x + \mathcal{A}y$ и $\mathcal{A}(\lambda x) = \lambda\mathcal{A}x$

    (!) $\displaystyle \forall y \in \Real^3 \ \exists! x \in \Real^3 \quad \mathcal{A}x = y =
    \begin{pmatrix}
        d \\ e \\ f
    \end{pmatrix}$

    $\displaystyle \begin{pmatrix}
         -\frac{8}{15}a                    \\
         -\frac{4}{3}a + \frac{4}{3}b - 4c \\
         \frac{2}{5}a + \frac{2}{3}c
    \end{pmatrix} =
    \begin{pmatrix}
        d \\ e \\ f
    \end{pmatrix} \Longleftrightarrow
    \begin{cases}
        a = -\frac{15}{8}d \\
        b = \frac{3}{4}e + 3c + a \\
        c = \frac{3}{2}f - \frac{3}{5}a
    \end{cases} \Longleftrightarrow
    \begin{cases}
        a = -\frac{15}{8}d \\
        b = \frac{3}{4}e + \frac{9}{2}f + \frac{3}{2}d \\
        c = \frac{3}{2}f + \frac{9}{8}d
    \end{cases} \Longleftarrow \exists!$ обратная

    $\Longleftarrow \mathcal{A}$ - линейный оператор, $\mathcal{A}: L \to L$

    Тогда его матрица $\displaystyle A =
    \begin{pmatrix}
        -\frac{8}{15} & 0           & 0           \\
        -\frac{4}{3}  & \frac{4}{3} & - 4         \\
        \frac{2}{5}   & 0           & \frac{2}{3}
    \end{pmatrix}$ - общий тип

    \item Решим вековое уравнение

    $\displaystyle |A - \lambda I| =
    \begin{vmatrix}
        -\frac{8}{15} - \lambda & 0                     & 0                     \\
        -\frac{4}{3}            & \frac{4}{3} - \lambda & - 4                   \\
        \frac{2}{5}             & 0                     & \frac{2}{3} - \lambda
    \end{vmatrix} = \left(-\frac{8}{15} - \lambda\right) \left(\frac{4}{3} - \lambda\right) \left(\frac{2}{3} - \lambda\right) =
    \left(\frac{8}{15} + \lambda\right) \left(\frac{4}{3} - \lambda\right) \left(\frac{2}{3} - \lambda\right) = 0$

    $\displaystyle \begin{cases}
         \lambda_1 = -\frac{8}{15} \\
         \lambda_2 = \frac{2}{3}  \\
         \lambda_3 = \frac{4}{3}
    \end{cases} \Longrightarrow $ диагональная форма $\displaystyle
    \begin{pmatrix}
        -\frac{8}{15} & 0           & 0           \\
        0             & \frac{2}{3} & 0           \\
        0             & 0           & \frac{4}{3}
    \end{pmatrix}$

    Матрица из собственных векторов (жорданов базис): $\displaystyle \begin{pmatrix}
        3 & 0 & 0 \\
        0 & 6 & 1 \\
        1 & 1 & 0
    \end{pmatrix}$
\end{enumerate}

